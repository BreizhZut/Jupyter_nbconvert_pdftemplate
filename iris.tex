



    
    
\chapter[Classification in the Iris data set]{Classification in the Iris data set\\ A test example for {\tt Jupyter nbconvert}}
    	  \epigraph{It is a million to one chances but it might just work.}{Dylan P. Tweed}

    
    
\begin{center}
\parbox{\abstractwidth}{\it
This document is the result of a {\tt nbconvert --to pdf} test on a very simple {\tt ipython notebook}. Along with this abstract, title, author, affiliations, dates and keywords metadata are displayed. Furthermore this document should illustrate additional features; captions, labels, bibliography figures and tables rescaling, cells hiding. The orignal notebook was build so that to test multiple templates (see \href{https://github.com/BreizhZut/Jupyter_nbconvert_pdftemplate.git}{github.com/BreizhZut/Jupyter\_nbconvert\_pdftemplate}) corresponding to documentation, article, book and presentations formats.
}
\end{center}
\par\vspace{\abstractsep}%\end{@twocolumnfalse}]     
    
    








\section{Data}\label{data}

The iris dataset is common test example for machine learning and can be
found in the \texttt{datasets} packages of \texttt{R} or as in this
instance the \texttt{sklearn} package in \texttt{python}. This data set
was first published in \cite{Fisher1936}, in was further use for the
purpose of testing machine learning classification algorithm such as in
\cite{RoHart1973}, \cite{Dasarathy1980}.


\begin{enumerate}
\def\labelenumi{\arabic{enumi}.}
\tightlist
\item
  Number of Instances: 150 (50 in each of three classes)
\item
  Number of Attributes: 4 numeric, predictive attributes and the class

  \begin{itemize}
  \tightlist
  \item
    sepal length in cm
  \item
    sepal width in cm
  \item
    petal length in cm
  \item
    petal width in cm
  \end{itemize}
\item
  class:

  \begin{itemize}
  \tightlist
  \item
    Iris-Setosa
  \item
    Iris-Versicolour
  \item
    Iris-Virginica
  \end{itemize}
\end{enumerate}


\subsection{Data frames}\label{data-frames}

The 3 class are indicated in the data as integers 0, 1 and 2:


    	 \begin{coding}
>>> \PY{c+c1}{\PYZsh{} This should appear everywhere}
... \PY{n}{Counter}\PY{p}{(}\PY{n}{target}\PY{p}{)}
Counter({0: 50, 1: 50, 2: 50})
\end{coding}
With the corresponding class names:


    	 \begin{coding}
>>> \PY{c+c1}{\PYZsh{} This should appear everywhere}
... \PY{n+nb}{list}\PY{p}{(}\PY{n}{target\PYZus{}names}\PY{p}{)}
['setosa', 'versicolor', 'virginica']
\end{coding}
We explore the first few element of the iris data set for each class:

\begin{itemize}
\tightlist
\item
  setosa encoded as 0 (see Table \ref{tab:hseto}),
\item
  versicolor encoded as 1 (see Table \ref{tab:hvers})
\item
  virginica encoded as 2 (see Table \ref{tab:hvirg}).
\end{itemize}

We note that the row are ordered by class. This is not important here,
since we try to test reference to some tables but for machine learning
tasks it is advised to shuffle the row both in the data and the target.


    	 \begin{table}

\centering
\adjustbox{max size={\textwidth}{\textheight}}{\begin{tabular}{lrrrr}
\toprule
{} &  sepal length &  sepal width &  petal length &  petal width \\
\midrule
0 &           5.1 &          3.5 &           1.4 &          0.2 \\
1 &           4.9 &          3.0 &           1.4 &          0.2 \\
2 &           4.7 &          3.2 &           1.3 &          0.2 \\
3 &           4.6 &          3.1 &           1.5 &          0.2 \\
4 &           5.0 &          3.6 &           1.4 &          0.2 \\
5 &           5.4 &          3.9 &           1.7 &          0.4 \\
6 &           4.6 &          3.4 &           1.4 &          0.3 \\
7 &           5.0 &          3.4 &           1.5 &          0.2 \\
8 &           4.4 &          2.9 &           1.4 &          0.2 \\
9 &           4.9 &          3.1 &           1.5 &          0.1 \\
\bottomrule
\end{tabular}
}

\caption{First ten rows corredsponding to the Setosa class}\customlabel{tab:hseto}{toto}
\end{table}

    	 \begin{table}

\centering
\adjustbox{max size={\textwidth}{\textheight}}{\begin{tabular}{lrrrr}
\toprule
{} &  sepal length &  sepal width &  petal length &  petal width \\
\midrule
50 &           7.0 &          3.2 &           4.7 &          1.4 \\
51 &           6.4 &          3.2 &           4.5 &          1.5 \\
52 &           6.9 &          3.1 &           4.9 &          1.5 \\
53 &           5.5 &          2.3 &           4.0 &          1.3 \\
54 &           6.5 &          2.8 &           4.6 &          1.5 \\
55 &           5.7 &          2.8 &           4.5 &          1.3 \\
56 &           6.3 &          3.3 &           4.7 &          1.6 \\
57 &           4.9 &          2.4 &           3.3 &          1.0 \\
58 &           6.6 &          2.9 &           4.6 &          1.3 \\
59 &           5.2 &          2.7 &           3.9 &          1.4 \\
\bottomrule
\end{tabular}
}

\caption{First ten rows corresponding to the Versicolor class}\customlabel{tab:hvers}{toto}
\end{table}

    	 \begin{table}

\centering
\adjustbox{max size={0.5\columnwidth}{\textheight}}{\begin{tabular}{lrrrr}
\toprule
{} &  sepal length &  sepal width &  petal length &  petal width \\
\midrule
100 &           6.3 &          3.3 &           6.0 &          2.5 \\
101 &           5.8 &          2.7 &           5.1 &          1.9 \\
102 &           7.1 &          3.0 &           5.9 &          2.1 \\
103 &           6.3 &          2.9 &           5.6 &          1.8 \\
104 &           6.5 &          3.0 &           5.8 &          2.2 \\
105 &           7.6 &          3.0 &           6.6 &          2.1 \\
106 &           4.9 &          2.5 &           4.5 &          1.7 \\
107 &           7.3 &          2.9 &           6.3 &          1.8 \\
108 &           6.7 &          2.5 &           5.8 &          1.8 \\
109 &           7.2 &          3.6 &           6.1 &          2.5 \\
\bottomrule
\end{tabular}
}

\caption{First ten rows corresponding to the Virginica class}\customlabel{tab:hvirg}{toto}
\end{table}
This text is a Lorem Ipsum, it should not appear in the documentation
template, and should add a lorem ipsum in the chapter and article
template. It should also appear in the beamer, to test the animation on
the table above.\par
\lipsum[1-4]

\subsection{Visualization}\label{visualization}

\begin{figure}

\centering
\adjustimage{max size={\textwidth}{0.65\textheight}}{iris_files/iris_21_0.pdf}

\caption{Scatter plot sepal width as a function of the sepal lenght for the iris dataset. As the legend indicates, the color code corresponds to the class.}\customlabel{fig:scat0}{toto}
\end{figure}
This text is once again a Lorem Ipsum, it should appear \textbf{only in
the article and chapter template}.\par
\lipsum[5-6]

\section{Model}\label{model}

For fun were testing different classification models for the iris
dataset using the Support Vector Classification (SVC) method. This
exemple is taken from the\texttt{sklearn} documantation. We test the SVC
methods with:

\begin{itemize}
\tightlist
\item
  a linear kernel (see Figure \ref{fig:svm_lin})
\item
  a Radial Basis Function kernel (RBF, see Figure \ref{fig:svm_rbf})
\item
  a degree 3 polynomial kernel (see Figure \ref{fig:svm_poly})
\end{itemize}



\subsection{Linear kernel SVC}\label{linear-kernel-svc}

\begin{figure}

\centering
\adjustimage{max size={0.5\columnwidth}{\textheight}}{iris_files/iris_28_0.pdf}

\caption{Same as Figure \ref{fig:scat0}. The shaded region correspond to the predictions of Linear SVC model.}\customlabel{fig:svm_lin}{toto}
\end{figure}
This text is a Lorem Ipsum, it should not appear in the documentation
template, and should add a lorem ipsum in the chapter and article
template. It should also appear in the beamer, to test the animation on
the table above.\par
\lipsum[7-8]

\subsection{Radial basis function kernel
SVC}\label{radial-basis-function-kernel-svc}

\begin{figure}

\centering
\adjustimage{max size={0.75\columnwidth}{\textheight}}{iris_files/iris_31_0.pdf}

\caption{Same as Figure \ref{fig:scat0}. The shaded region correspond to the predictions of SVC RBF model.}\customlabel{fig:svm_rbf}{toto}
\end{figure}
This text is once again a Lorem Ipsum, it should appear \textbf{only in
the article and chapter template}.\par
\lipsum[9-10]

\subsection{Polynomial kernel SVC}\label{polynomial-kernel-svc}

\begin{figure}

\centering
\adjustimage{max size={\textwidth}{\textheight}}{iris_files/iris_34_0.pdf}

\caption{Same as Figure \ref{fig:scat0}. The shaded region correspond to the predictions of polynomial SVC model.}\customlabel{fig:svm_poly}{toto}
\end{figure}
This text is a Lorem Ipsum, it should appear \textbf{only in the article
and chapter template}.\par
\lipsum[1-3]




    
    
